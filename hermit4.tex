\documentclass[12pt]{article}
\pagestyle{plain}
\setlength {\topmargin} {-1.0cm}
\setlength {\textwidth} {16.0cm}
\setlength {\textheight} {22.5cm}
\setlength {\oddsidemargin} {0cm}
\begin{document}

%  Useful newcommands
%  @@@@@@@@@@@@@@@@@@

\newcommand{\etaI}{\eta_{\scriptscriptstyle I}}
\newcommand{\etaR}{\eta_{\scriptscriptstyle R}}
\newcommand{\etaU}{\eta_{\scriptscriptstyle U}}
\newcommand{\BH}{\scriptscriptstyle BH}
\newcommand{\GR}{\scriptscriptstyle GR}
\newcommand{\barrho}{\mathop{<\mskip -4mu\rho\mskip -4mu>}}
\newcommand{\brho}{\mbox{\boldmath $\rho $}} %due to Sue Cowell 9/99
\newcommand{\bom}{\mbox{\boldmath $\omega $}}
\newcommand{\bsig}{\mbox{\boldmath $\sigma $}}
\newcommand\pminus{\phantom{-}}
\newcommand{\etal}{{\it et al.\ }}
\newcommand{\cm}{c.m.\ }
\newcommand{\ie}{i.e.\ }
\newcommand{\cf}{cf.\ }
\newcommand{\rn}[1]{(\ref{#1})}
\newcommand{\be}{\begin{equation}}
\newcommand{\ee}{\end{equation}}
\newcommand{\bea}{\begin{eqnarray}}
\newcommand{\eea}{\end{eqnarray}}
\newcommand{\bean}{\begin{eqnarray*}}
\newcommand{\eean}{\end{eqnarray*}}
\newcommand{\bdis}{\begin{displaymath}}
\newcommand{\edis}{\end{displaymath}}
\newcommand{\half}{{\textstyle \frac{1}{2}}}
\newcommand{\third}{{\textstyle \frac{1}{3}}}
\newcommand{\quart}{{\textstyle \frac{1}{4}}}
\newcommand{\sixth}{{\textstyle \frac{1}{6}}}
\newcommand{\tenth}{{\textstyle \frac{1}{10}}}
\newcommand{\twelfth}{{\textstyle \frac{1}{12}}}
\newcommand{\sixteenth}{{\textstyle \frac{1}{16}}}
\newcommand{\twothird}{{\textstyle \frac{2}{3}}}
\newcommand{\eigth}{{\textstyle \frac{1}{8}}}
\newcommand{\onetfour}{{\textstyle \frac{1}{24}}}
\newcommand{\onehtw}{{\textstyle \frac{1}{120}}}
\newcommand{\thalf}{{\textstyle \frac{3}{2}}}
\newcommand{\fhalf}{{\textstyle \frac{5}{2}}}
\newcommand{\fthird}{{\textstyle \frac{5}{3}}}
\newcommand{\fat}[1]{{\bf #1}}
\newcommand{\doo}[2]{{\frac{\partial #1}{\partial #2}}}
\newcommand{\ex}{{\rm e}}
\newcommand{\bc}{\begin{center}}
\newcommand{\ec}{\end{center}}
\newcommand{\hb}{\hfill\break}
\newcommand{\bi}{\begin{itemize}}
\newcommand{\ei}{\end{itemize}}
\newcommand{\kms}{{\rm km}\,{\rm s}^{-1}}
\newcommand{\pcc}{{\rm pc}^{-3}}
\newcommand{\bt}{\begin{tabbing}}
\newcommand{\et}{\end{tabbing}}
\newcommand{\vinf}{v_\infty}
\newcommand{\kmsec}{km\,${\rm s}^{-1}$}
\newcommand{\kmsecc}{km\,${\rm s}^{-1}\:$}
\def\ZZ#1{$\scriptstyle #1$}
\def\plusplus{\raise 0.3ex\hbox{${\scriptstyle ++}$}{}}

\def\AJ{{\it Astron. J.} }
\def\APJ{{\it Astrophys. J.} }
\def\ApJ{{\it Astrophys. J.} }
\def\MN{{\it Mon. Not. R. Astron. Soc.} }
\def\AA{{\it Astron. Astrophys.} }
\def\CM{{\it Celes. Mech.} }
\def\CMD{{\it Celes. Mech. Dyn. Ast.} }
\def\PASJ{{\it Publ. Astron. Soc. Japan} }
\def\SJA{S.J. Aarseth}
\def\SAA{S.J. Aarseth }

%%%%%%%%%%%%%%%%%%%%%%%%%%%%%%%%%%%%%%%%%%%%%
% Alison's definitions of bold \rho and \mu (three subscript levels)

\font\elevenmib=cmmib10 at 11pt \skewchar\elevenmib='177
\font\eightmib=cmmib10 at 8pt   \skewchar\eightmib='177
\font\sixmib=cmmib10 at 6pt     \skewchar\sixmib='177
%
\newfam\mibfam
\textfont\mibfam=\elevenmib
\scriptfont\mibfam=\eightmib
\scriptscriptfont\mibfam=\sixmib

\def\hexnum#1{\ifnum#1<10 \number#1\else
 \ifnum#1=10 A\else\ifnum#1=11 B\else\ifnum#1=12 C\else
 \ifnum#1=13 D\else\ifnum#1=14 E\else\ifnum#1=15 F\fi\fi\fi\fi\fi\fi\fi}
\def\mib{\hexnum\mibfam}

\mathchardef\bmu="0\mib16

%\setlength {\topmargin} {-1.0cm}
%\setlength {\textwidth} {16.0cm}
%\setlength {\textheight} {22.5cm}
 
 
\centerline {\Large {\bf {HERMIT4 User Manual~~~~}}}
\bigskip
\centerline {\Large {Sverre Aarseth~~~}}
\bigskip
\centerline {\large {\tt Email: sverre@ast.cam.ac.uk}~~~~~}
\bigskip
\centerline {\large {Institute of Astronomy, University of Cambridge~~~~}}
\medskip
\bigskip
\section{Introduction}

This is a User Manual for {\tt HERMIT4}, a code that was developed
during February 2004.
It is a combination of two proven formulations which are studied by the
Hermite method.
Most particles are integrated by the time-symmetric scheme of
Kokubo, Yoshinaga \& Makino [1998].
In order to treat close two-body encounters more accurately, the so-called
Burdet--Heggie regularization method is employed.
For simplicity only one such encounter is treated.
This means that the relative motion must be hyperbolic and any rare
binaries with small periods are combined into one body as an inelastic
collision.
The time-symmetric scheme is based on the presence of one dominant
central body which allows iteration of the two-body term, thereby reducing
the systematic error considerably.
Constant (quantized) time-steps for small perturbations also enhance the 
numerical stability.
However, it is still a brute-force method.
Hence the {\tt CPU} requirements limit the maximum particle number to
a few hundred, depending on the desired time interval.

\section{Code structure}

The whole code consists of some 2265 lines including comments and layout
space.
It is written in {\tt FORTRAN} and is F77 compliant but also compiles
with Intel or F95 (ignore warning messages about non-standard expressions).
There are about 32 routines altogether, with mnemonic names of maximum six
characters (see Table 1).
Likewise, all the {\tt FORTRAN} statements are in upper case while the
comments are in lower case.
The coding conforms to a strict style and layout for clarity.

The main code relies on a general {\tt common} block called
{\tt commonp.h} which contains a number of arrays of maximum size {\tt NMAX}
defined as a parameter, as well as many useful scalars.
This enables a calculation to be split into several stages by saving all
the {\tt common} variables after a specified {\tt CPU} time (or by a
`{\tt touch STOP}' facility at arbitrary times), followed by a restart.
Except for some special situations, this gives rise to reproducible results
which are essential for experimental purposes as well as problem
investigation.

The code consists of three main parts: input, output and integration, with
the latter split into several routines dealing with two different methods.
Since the choice is between integration of single particles or one close
encounter, the decision-making is simplified and requires very little
overheads.
Several optional procedures are included for convenience (see routine
{\tt INPUT} for definitions).

\section{Getting started}

Once the file {\tt hermit4.tar.Z} is downloaded and uncompressed, the routines
are extracted by `{\tt tar xvf hermit4.tar}' and copied to directory
{\tt Hermit4}.
A test input template called {\tt input} is also included.

Before compiling, check the {\tt FORTRAN} directives in the {\tt Makefile},
\ie whether {\tt f77} or {\tt g77} and use the highest optimization level.
Compile by the command `{\tt make hermit4}' which should produce the
executable {\tt hermit4}.
One possible reason for failure could be the {\tt CPU} timer function
{\tt etime} in routine {\tt CPUTIM} which is system dependent.
Any other complaints and strange run-time behaviour should be reported after
making every effort to ascertain the problem.
For frequent usage it is recommended to create separate working directories.
It is also a good idea to save the original files when making changes.

To start a test run, place the executable and a template input file
{\tt input} in one directory and type the command
`{\tt hermit4 $<$ input $>$ output \&}'.
Although the results are machine- and compiler-dependent, it is expected that
the run will finish normally.

The restart facility can be tested in the following way.
First specify option \#1 = 1 at input time.
This will ensure a {\tt common} save on unit {\tt fort.1} if the specified
{\tt CPU} time is exceeded or {\tt TIME $>=$ TCRIT}, with {\tt TCRIT} given
in the input file.
The calculation can then be continued from the {\tt common} save by
`{\tt hermit4 $<$ rs $>$} output2 \&',
using the restart input file {\tt rs} containing `{\tt KSTART TCOMP}'.
Here {\tt KSTART = 1} denotes a new run (followed by the required input)
or $= 2$ for standard restart, and {\tt TCOMP} is the {\tt CPU} time in
minutes.
If the calculation is terminated at {\tt TIME = TCRIT} with \#2 = 2, 
{\tt fort.2} must be copied to {\tt fort.1} before the restart, in which
case a new value of {\tt TCRIT} needs to be prescribed.
Also note the possibility of reading some new parameters at restart if
{\tt KSTART $>$ 2}.
In the current version only one option can be changed but this can readily
be generalized.

\section{Input parameters}

A standard input file consists of four lines defining the {\tt CPU} time,
membership, dimensionless time-step factor {\tt ETA}, options and two-body
regularization parameters.
The overall accuracy is controlled by {\tt ETA}, which enters the time-step
criterion in a square root.
The appropriate choice depends on many factors and is best decided
after extensive testing but for practical work a value around 0.003 seems
adequate.
Likewise, the options can be selected by consulting routine {\tt INPUT}.
Hence only a few values need to be changed in the input template once the
first calculation has been performed.

The question of the initial particle distribution must be decided by the
user and only a few light bodies have been included for code testing
purposes.
It is recommended to generate initial masses, coordinates and velocities
by a separate program, to be read in by routine {\tt DATA}.
Note that the calculations are performed in heliocentric coordinates,
with unit mass assumed for the massive central body.
The code is not suitable for studying high eccentricities because the
indirect terms may be large, giving rise to unfavourable fluctuations.
For central distances well outside the unit length, ensure that the
maximum time-step {\tt DTMAX} (set in {\tt DATA}) is sufficiently large.

Input parameters for regularization should be consistent with the
particle distribution.
Thus the proper value of the close encounter time-step, {\tt DTMIN},
should match the critical two-body separation, {\tt RMIN}.
Because the solar perturbation also plays a role (and its effect is
included), it is recommended to use a conservative value of {\tt RMIN}
much smaller than the Hill (or Roche) radius.
Since most encounters treated will be hyperbolic, the choice of the
regularized time-step parameter {\tt ETAU} is less important.
For a binary, there are $2 \pi/${\tt ETAU} steps per orbit, while a
constant {\tt DTAU} determined at {\tt R = RMIN} is appropriate for
hyperbolic motion.
In practice, the regularization step counter should be much less than
the number of standard integration steps.

Since planetary systems are intended, the output time interval and
termination time, {\tt DELTAT} and {\tt TCRIT}, are specified in years
as input.
Scaling the time to other systems, such as a planet with satellites,
can readily be made in terms of the square root of the solar mass
ratio.

The code allows for integration of an additional component of mass-less
particles.
This is achieved by first allocating {\tt NMASS} massive planetesimals 
in the sequential arrays, followed by {\tt N - NMASS} particles of
zero mass.
Hence all particles are advanced in the same way, with the force
summation restricted to {\tt NMASS}.
At present, {\tt NMASS = N} and this feature has not been fully tested.

\section{Implementation}

The integration of single particles employs the time-symmetric Hermite
block-step scheme [Kokubo \etal 1998].
At each cycle, a list of particles due to be advanced is determined.
The perturbations due to other planetesimals is evaluated separately
from the dominant two-body term.
This allows an extra iteration after the corrector has been applied which
results in a significant accuracy gain (as can be seen by suppressing
the iteration loop).

Close encounters are characterized by small time-steps.
A search is therefore made for the dominant neighbour if the time-step is
below {\tt DTMIN}.
The candidate particle is accepted if {\tt R $<$ RMIN} and the relative velocity
is negative.
Masses, coordinates and velocities for the two particles are then placed in
the two first locations of the respective arrays.
Relevant quantities for the new pair are initialized according to the equations
of motion [cf. Aarseth 2003] and the centre of mass (c.m.) is introduced as a
fictitious particle at location {\tt NTOT = N + 1}.
Finally, the force polynomial and time-step for the c.m. are evaluated in
the single particle approximation with reduced time-step.

The subsequent integration of the c.m. particle is formally similar to that
of single particles, except that the force is obtained by vectorial
summation over the components unless the c.m. approximation applies.
Note that since heliocentric coordinates are used, the so-called indirect
force terms must be included in all the equations of motion.
Conversely, the force on single particles due to a regularized pair is
treated more carefully within a factor ($\simeq 100$) of the two-body
separation.
This entails inverting the regularized time interval by a Taylor series
expansion and transforming to physical coordinates and velocities before
summing over the individual components.
Hence the direct force loop starts at {\tt IFIRST = 2*NPAIRS + 1},
where {\tt NPAIRS = 1} during regularization, in which case the
corresponding contribution is added afterwards.

The equations of motion for two-body regularization are initialized by
boot-strapping, yielding explicit derivatives and also advanced by
the Hermite integration scheme.
Each regularized time-step is converted to physical units by a high-order
Taylor series based on known coefficients.
Hyperbolic regularizations are terminated when the initial separation is
exceeded.
In order to improve the accuracy and permit larger new quantized time-steps,
the relative motion is advanced up to the end of the next block-step.
This is followed by standard initialization of polynomials and time-steps
for both components.

Collisions are defined by {\tt R $<$ RCOLL}, where {\tt RCOLL} is currently
defined by a data statement in routine {\tt BHINT} (but can readily be
included as input, provided it is added in {\tt commonp.h}).
For practical reasons, the implementation of a collision (option \#7) is
again delayed until the end of the block-step, leaving the c.m. as a new
particle after removal of the components.

Particles are removed (option \#9) from the calculation at output time for
two different conditions.
This occurs if the eccentricity exceeds a critical value (defined by
{\tt ECRIT} in routine {\tt OUTPUT}) and the central distance is suitably
large (\ie {\tt RESC}) or the orbital eccentricity is less than 1.
The latter case deals with sun-grazing orbits which would otherwise need
many small time-steps.
Other criteria may be introduced to suit the requirements.

Eccentric binaries are treated as close encounters in the pericentre region.
However, in order to restrict the treatment to just one regularization for
simplicity, the components of binaries with period below {\tt TMERGE}
(in days, defined in routine {\tt BHINT}) are combined inelastically as
for physical collisions.
Note that the critical period must at least be large enough to include a
circular binary of semi-major axis {\tt RMIN}, otherwise an inconsistency
may arise.
In fact, this requirement needs to be satisfied by a considerable margin
in order to handle hyperbolic encounters at the appropriate times.

Note that the calculation can be terminated at any time by typing
`touch STOP' whereupon a {\tt common} save will occur on {\tt fort.1},
from which a restart can be made.


\section*{References}

\medskip
\noindent
Aarseth, S.J. [2003], {\it Gravitational N-Body Simulations}
(Cambridge University Press).

\medskip
\noindent
Burdet, C.A. [1967], `Regularization of the two-body problem',
{\it Z. Angew. Math. Phys.} {\bf 18}, 434--438.

\medskip
\noindent
Heggie, D.C. [1973], `Regularization using a time-transformation only',
in {\it Recent Advances in Dynamical Astronomy}, ed. B.D. Tapley \&
V. Szebehely (Reidel, Dordrecht), 34--37.

\medskip
\noindent
Kokubo, E., Yoshinaga, K. \& Makino, J. [1998], `On a time-symmetric
Hermite integrator for planetary N-body simulation',
\MN {\bf 297}, 1067--1072.

\newpage
\bigskip
\section{Appendix}
Table 1 contains a list of all the subroutines, the main calling structure
and a brief description of their purpose.
The subroutine names are constructed in analogy with the large $N$-body
codes.

% \newpage
\begin{table}[h]
\vspace*{-5mm}
\centering
\caption{{\it Subroutine definitions.}}
\label{routines}
\begin{tabular}{lll}
\hline\hline
\multicolumn{1}{l}{Routine} &
\multicolumn{1}{l}{Called by} &
\multicolumn{1}{l}{Description} \\
\hline\hline
{\tt BHINIT} & {\tt BHREG} &Initialization of two-body regularization. \\
{\tt BHINT}  & {\tt INTGRT} &Integration of two-body motion. \\
{\tt BHLIST} & {\tt BHINIT} &Perturber selection. \\
{\tt BHPERT} & {\tt BHINIT} \& BHINT &Perturbing forces. \\
{\tt BHREG}  & {\tt HERMIT4} &New Burdet-Heggie regularization. \\
{\tt BHTERM} & {\tt HERMIT4} &Termination of regularization. \\
{\tt BLOCK}  & &Block data initialization. \\
{\tt BODIES} & {\tt OUTPUT} &Output of single particles and binaries. \\
{\tt CMF}    & {\tt NBINT}  &Force on c.m. particle. \\
{\tt CPUTIM} & {\tt INTGRT} \& {\tt OUTPUT} &Elapsed {\tt CPU} time in minutes. \\
{\tt DATA}   & {\tt START}  &Generation of initial conditions. \\
{\tt ENERGY} & {\tt OUTPUT} \& {\tt BHTERM} &Evaluation of the total energy. \\
{\tt FPERT}  & {\tt SEARCH} &Perturbing force on dominant bodies. \\
{\tt FPOLY1} & {\tt START}  \& {\tt BHINIT} &Initialization of force polynomial. \\
{\tt HERMIT4} & &Master routine. \\
{\tt IBLOCK} & {\tt START} &Initialization of block steps. \\
{\tt INPUT}  & {\tt START} &Parameter input. \\
{\tt INTGRT} & {\tt HERMIT4} &Main integrator flow control. \\
{\tt MYDUMP} & {\tt INTGRT} \& {\tt OUTPUT} &{\tt COMMON} save or read. \\
{\tt NBINT}  & {\tt INTGRT} &N-body integration with iteration. \\
{\tt OUTPUT} & {\tt HERMIT4} &Energy check, output and escape removal. \\
{\tt RAN2}   & {\tt DATA}   &Random number generator. \\
{\tt REMOVE} & {\tt BHINT} \& {\tt OUTPUT} &Particle removal. \\
{\tt RESOLV} & {\tt BHINT} \& {\tt NBINT}  &Two-body coordinates and velocities. \\
{\tt SEARCH} & {\tt NBINT}  &Close encounter search. \\
{\tt START}  & {\tt HERMIT4} &Initialization of data and polynomials. \\
{\tt STEPI}  & {\tt NBINT}  &Fast time-step expression (option \#5). \\
{\tt STEPK}  & {\tt STEPS}  &Truncation to commensurate block-step. \\
{\tt STEPS}  & {\tt START} \& {\tt BHINIT} &Initialization of time-steps. \\
{\tt TSTEP}  & {\tt NBINT}  &Standard time-step criterion. \\
{\tt XVPRED} & {\tt OUTPUT} \& {\tt BHINT} &Prediction of coordinates and velocities. \\
{\tt ZERO}   & {\tt START}  &Initialization of global scalars. \\
\hline\hline
\end{tabular}
\end{table}

\end{document}
